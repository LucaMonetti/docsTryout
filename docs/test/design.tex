\PassOptionsToPackage{unicode=true}{hyperref} % options for packages loaded elsewhere
\PassOptionsToPackage{hyphens}{url}
%
\documentclass[]{TWDocumentFull}
\usepackage{lmodern}
\usepackage{amssymb,amsmath}
\usepackage{ifxetex,ifluatex}
\usepackage{fixltx2e} % provides \textsubscript
\ifnum 0\ifxetex 1\fi\ifluatex 1\fi=0 % if pdftex
  \usepackage[T1]{fontenc}
  \usepackage[utf8]{inputenc}
  \usepackage{textcomp} % provides euro and other symbols
\else % if luatex or xelatex
  \usepackage{unicode-math}
  \defaultfontfeatures{Ligatures=TeX,Scale=MatchLowercase}
\fi
% use upquote if available, for straight quotes in verbatim environments
\IfFileExists{upquote.sty}{\usepackage{upquote}}{}
% use microtype if available
\IfFileExists{microtype.sty}{%
\usepackage[]{microtype}
\UseMicrotypeSet[protrusion]{basicmath} % disable protrusion for tt fonts
}{}
\IfFileExists{parskip.sty}{%
\usepackage{parskip}
}{% else
\setlength{\parindent}{0pt}
\setlength{\parskip}{6pt plus 2pt minus 1pt}
}
\usepackage{hyperref}
\hypersetup{
            pdftitle={Web Design},
            pdfauthor={Luca Monetti},
            pdfborder={0 0 0},
            breaklinks=true}
\urlstyle{same}  % don't use monospace font for urls
\usepackage{longtable,booktabs}
% Fix footnotes in tables (requires footnote package)
\IfFileExists{footnote.sty}{\usepackage{footnote}\makesavenoteenv{longtable}}{}
\setlength{\emergencystretch}{3em}  % prevent overfull lines
\providecommand{\tightlist}{%
  \setlength{\itemsep}{0pt}\setlength{\parskip}{0pt}}
\setcounter{secnumdepth}{0}
% Redefines (sub)paragraphs to behave more like sections
\ifx\paragraph\undefined\else
\let\oldparagraph\paragraph
\renewcommand{\paragraph}[1]{\oldparagraph{#1}\mbox{}}
\fi
\ifx\subparagraph\undefined\else
\let\oldsubparagraph\subparagraph
\renewcommand{\subparagraph}[1]{\oldsubparagraph{#1}\mbox{}}
\fi

% set default figure placement to htbp
\makeatletter
\def\fps@figure{htbp}
\makeatother

\title{Titolo}

\editor{Carraro Agnese}
\editor{Marcon Giulia}
\editor{Pistori Gaia}
\editor{Vasquez Manuel Felipe}

\reviewer{Monetti Luca}
\reviewer{Piola Andrea}
\reviewer{Dal Bianco Riccardo}
\classification{Esterno}
\version{1.0}

% \renewcommand{\maketitle}{\frontmatter}

\renewcommand{\maketitle}{%
\newgeometry{left=3.5cm, right=3.5cm}% Ensure geometry is changed
\begin{titlepage}
    \begin{center}
        % Adjust images
        \raisebox{1cm}{\includegraphics[height=3cm]{logo_unipd.png}}%
        \hspace{0.5cm}%
        \raisebox{0cm}{\includegraphics[height=5cm]{logo.png}}%

        \vspace{1cm}
        \small\textcolor{darkgray}{techwave.unipd@gmail.com}

        \vspace{0.5cm}
        \huge\textbf{\@title}%
    \end{center}

    \vspace{2cm}

    % Document details - use proper spacing and ensure macros are defined
    \noindent\parbox{\textwidth}{\@labeltext{Redatto da:} \@LabelValue{\editorlist}}%

    \vspace{0.5em}

    \noindent\parbox{\textwidth}{\@labeltext{Revisionato da:} \@LabelValue{\reviewerlist}}%

    \vspace{0.5em}

    \noindent\parbox{\textwidth}{\@labeltext{Durata riunione:} \@LabelValue{\@duration}}%

    \vspace{0.5em}

    \noindent\parbox{\textwidth}{\@labeltext{Classificazione:} \@LabelValue{\@classification}}%

    \vspace{0.5em}

    \noindent\parbox{\textwidth}{\@labeltext{Versione:} \@LabelValue{\@version}}%
\end{titlepage}
\restoregeometry % Restore original geometry
}

\title{Web Design}
\author{Luca Monetti}
\date{}

\begin{document}
\maketitle

\hypertarget{design}{%
\section{Design}\label{design}}

\hypertarget{architettura-dellinformazione}{%
\subsection{Architettura
dell'informazione}\label{architettura-dellinformazione}}

Come si organizzano, archiviano e ricercano Informazioni.

Elementi principali:

\begin{itemize}
\tightlist
\item
  Contesto: Fondamentale. Può modificare l'aspettativa dell'utente.
\item
  Contenuto
\item
  Utenti
\end{itemize}

L'utente non deve sentirsi \textbf{perso} all'interno del sito:

\begin{longtable}[]{@{}ll@{}}
\toprule
Domanda & Risposta\tabularnewline
\midrule
\endhead
Dove sono? & Presenza di Breadcrumb\tabularnewline
\bottomrule
\end{longtable}

La presenza di una mappa del sito rappresenta una metrica: Se tante
persone accedono alla pagina il mio sito non è chiaro. Non dovrebbe
essere necessario accedere alla mappa per capire dove devo andare.

\textbf{Non è uno strumento di orientamento}

\hypertarget{problemi-dellorganizzazione-dellinformazione}{%
\subsubsection{Problemi dell'organizzazione
dell'informazione}\label{problemi-dellorganizzazione-dellinformazione}}

Possibile utilizzare diversi schemi, alcuni ambigui, altri meno.

\begin{itemize}
\tightlist
\item
  \textbf{Schemi Metaforici}: Possono essere più o meno comprendibili.
\item
  \textbf{Strutture Organizzative}: Le principali sono 4:

  \begin{itemize}
  \tightlist
  \item
    Sequenza
  \item
    Gerarchia
  \item
    A matrice
  \item
    Ipertesto
  \end{itemize}
\end{itemize}

\hypertarget{legge-di-hick}{%
\paragraph{Legge di HIck}\label{legge-di-hick}}

Il tempo necessario per compiere una scelta non dipende dal numero di
scelte ma da come sono presentate.

\begin{itemize}
\tightlist
\item
  Se rieco ad organizzare il menù secondo un ordine comodo per l'utente
  strutture ampie sono preferibili rispetto a quelle profonde
\item
  Altrimenti:

  \begin{itemize}
  \tightlist
  \item
    Più livelli coerenti
  \item
    Personalizzare il menù mostrando solo una parte delle scelte
    possibili.
  \end{itemize}
\end{itemize}

\hypertarget{comportamento-degli-utenti}{%
\subsection{Comportamento degli
utenti}\label{comportamento-degli-utenti}}

Caso migliore:

\begin{enumerate}
\def\labelenumi{\arabic{enumi}.}
\tightlist
\item
  L'utente pone una domanda
\item
  Ricerca / Naviga
\item
  L'utente riceve una risposta
\item
  Fine della ricerca
\end{enumerate}

Gli utenti spesso non sanno cosa vogliono. Terminano spesso la ricerca
con insuccesso.

\hypertarget{metafora-della-pesca}{%
\subsubsection{Metafora della pesca}\label{metafora-della-pesca}}

Gli utenti vengono paragonati a dei pescatori:

\begin{longtable}[]{@{}lll@{}}
\toprule
\begin{minipage}[b]{0.30\columnwidth}\raggedright
Nome\strut
\end{minipage} & \begin{minipage}[b]{0.30\columnwidth}\raggedright
Tipo Utente\strut
\end{minipage} & \begin{minipage}[b]{0.30\columnwidth}\raggedright
Risposta\strut
\end{minipage}\tabularnewline
\midrule
\endhead
\begin{minipage}[t]{0.30\columnwidth}\raggedright
Tiro perfetto\strut
\end{minipage} & \begin{minipage}[t]{0.30\columnwidth}\raggedright
L'utente sa cosa sta cercando\strut
\end{minipage} & \begin{minipage}[t]{0.30\columnwidth}\raggedright
Barra di ricerca\strut
\end{minipage}\tabularnewline
\begin{minipage}[t]{0.30\columnwidth}\raggedright
Trappola per aragoste\strut
\end{minipage} & \begin{minipage}[t]{0.30\columnwidth}\raggedright
L'utente non ha un'idea precisa, impara nel percoso. Processo
esplorativo\strut
\end{minipage} & \begin{minipage}[t]{0.30\columnwidth}\raggedright
Barra di navigazione. Chatbot.\strut
\end{minipage}\tabularnewline
\begin{minipage}[t]{0.30\columnwidth}\raggedright
Pesca con la rete\strut
\end{minipage} & \begin{minipage}[t]{0.30\columnwidth}\raggedright
L'utente vuole esplorare tutto all'interno di un argomento\strut
\end{minipage} & \begin{minipage}[t]{0.30\columnwidth}\raggedright
/\strut
\end{minipage}\tabularnewline
\begin{minipage}[t]{0.30\columnwidth}\raggedright
Boa di segnalazione\strut
\end{minipage} & \begin{minipage}[t]{0.30\columnwidth}\raggedright
L'utente vuole ritrovare un elemento\strut
\end{minipage} & \begin{minipage}[t]{0.30\columnwidth}\raggedright
Bookmark del browser. Wishlist.\strut
\end{minipage}\tabularnewline
\bottomrule
\end{longtable}

\end{document}
